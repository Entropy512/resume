% Andrew Dodd's Resume
% Created: 12 Jan 2005
% $Date: 2005-02-20 19:38:05 -0500 (Sun, 20 Feb 2005) $
% $Rev: 7 $

\documentclass[10pt,oneside]{article}
\usepackage{geometry}
\usepackage[T1]{fontenc}
\usepackage{enumitem}

\pagestyle{empty}
\geometry{letterpaper,tmargin=0.7in,bmargin=0.7in,lmargin=0.7in,rmargin=0.7in,headheight=0in,headsep=0in,footskip=.3in}

\setlength{\parindent}{0in}
\setlength{\parskip}{0in}
\setlength{\itemsep}{0in}
\setlength{\topsep}{0in}
\setlength{\tabcolsep}{0in}

% Name and contact information
\newcommand{\name}{Andrew Thomas Dodd}
\newcommand{\addra}{264 Lower Stella Ireland Road, Apt 1A}
\newcommand{\addrb}{Binghamton, NY 13905}
\newcommand{\addr}{\addra, \addrb}
\newcommand{\phone}{(607) 759-0996}
\newcommand{\email}{atdodd@gmail.com}
\newcommand{\github}{https://github.com/Entropy512/}

\hyphenation{Summer}

\makeatletter
\def\txtname{resume_plain}\edef\txtname{\expandafter\strip@prefix\meaning\txtname}
\edef\fname{\jobname}

\ifx\fname\txtname
%%%%%%%%%%%%%%%%%%%%%%%%%%%%%%%%%%%%%%%%%%%%%%%%%%%%%%%%%
% New commands and environments
% Stripped down version for text-only output

% This defines how the name looks
\newcommand{\bigname}[1]{
	\begin{center}\Huge\scshape#1\end{center}
}

% A ressection is a main section (<H1>Section</H1>)
\newenvironment{ressection}[1]{
	\vspace{4pt}
	{\Large#1}
	\begin{itemize}
	\vspace{3pt}
}{
	\end{itemize}
}

% A resitem is a simple list element in a ressection (first level)
\newcommand{\resitem}[1]{
	\vspace{-4pt}
	\item \begin{flushleft} #1 \end{flushleft}
}

% A ressubitem is a simple list element in anything but a ressection (second level)
\newcommand{\ressubitem}[1]{
	\vspace{-1pt}
	\item \begin{flushleft} #1 \end{flushleft}
}

% A resbigitem is a complex list element for stuff like jobs and education:
%  Arg 1: Name of company or university
%  Arg 2: Location
%  Arg 3: Title and/or date range
\newcommand{\resbigitem}[3]{
	\vspace{-5pt}
	\item
	\textbf{#1}---#2; \quad \textit{#3}
}

% This is a list that comes with a resbigitem
\newenvironment{ressubsec}[3]{
	\resbigitem{#1}{#2}{#3}
	\vspace{-2pt}
	\begin{itemize}
}{
	\end{itemize}
}

% This is a simple sublist
\newenvironment{reslist}[1]{
	\resitem{\textbf{#1}}
	\vspace{-5pt}
	\begin{itemize}
}{
	\end{itemize}
}


\else
%%%%%%%%%%%%%%%%%%%%%%%%%%%%%%%%%%%%%%%%%%%%%%%%%%%%%%%%%
% New commands and environments

% This defines how the name looks
\newcommand{\bigname}[1]{
	\begin{center}\fontfamily{phv}\selectfont\Huge\scshape#1\end{center}
}

% A ressection is a main section (<H1>Section</H1>)
\newenvironment{ressection}[1]{
	\vspace{3pt}
	{\fontfamily{phv}\selectfont\Large#1}
	\begin{itemize}[leftmargin=12pt]
	\vspace{2pt}
}{
	\end{itemize}
}

% A resitem is a simple list element in a ressection (first level)
\newcommand{\resitem}[1]{
	\vspace{-4pt}
	\item \begin{flushleft} #1 \end{flushleft}
}

% A ressubitem is a simple list element in anything but a ressection (second level)
\newcommand{\ressubitem}[1]{
	\vspace{-1pt}
	\item \begin{flushleft} #1 \end{flushleft}
}

% A resbigitem is a complex list element for stuff like jobs and education:
%  Arg 1: Name of company or university
%  Arg 2: Location
%  Arg 3: Title and/or date range
\newcommand{\resbigitem}[2]{
	\vspace{-5pt}
	\item
	\textbf{#1}---\textit{#2}
}

% This is a list that comes with a resbigitem
\newenvironment{ressubsec}[2]{
	\resbigitem{#1}{#2}
	\vspace{-2pt}
	\begin{itemize}[leftmargin=12pt]
	}{
	\end{itemize}
}

% This is a simple sublist
\newenvironment{reslist}[1]{
	\resitem{\textbf{#1}}
	\vspace{-5pt}
	\begin{itemize}[leftmargin=12pt]
}{
	\end{itemize}
}


\fi
%%%%%%%%%%%%%%%%%%%%%%%%%%%%%%%%%%%%%%%%%%%%%%%%%%%%%%%%%
% Now for the actual document:

\begin{document}

\ifx\fname\txtname
\else
\fontfamily{ppl} \selectfont
\fi

% Name with horizontal rule
\bigname{\name}

\ifx\fname\txtname
\else
\vspace{-8pt} \rule{\textwidth}{1pt}
\fi

\ifx\fname\txtname
\vspace{-1pt} {\small\itshape \addra

\addrb

\phone

\email

\github}

\else
\vspace{-1pt} {\small\itshape \addr \hfill \phone; \email}

\vspace{1pt} {\small\itshape \hfill \github}
\fi

\vspace{8 pt}


%%%%%%%%%%%%%%%%%%%%%%%%%
\begin{ressection}{Objective}
	\resitem{Currently seeking employment in electronics hardware design and/or embedded software}
\end{ressection}

\begin{ressection}{Education}

	\begin{ressubsec}{Rutgers University}{M.S. in Electrical and Computer Engineering, December 2005}
		\ressubitem{Concentration in Digital Signal Processing and Wireless Communications}
	\end{ressubsec}

	\begin{ressubsec}{Cornell University College of Engineering}{Bachelor of Science in Electrical Engineering, May 2002}
		\ressubitem{Concentration in RF Design and Wireless Communications}
	\end{ressubsec}

\end{ressection}

\begin{ressection}{Work Experience}

	\begin{ressubsec}{The Raymond Corporation}{December 2016 - November 2023}
		\ressubitem{Primary maintainer of obstruction detection subsystem for Raymond Courier AGVs - configuration of SICK FlexiSoft safety controller logic and SICK S300 safety LiDAR configuration}
		\ressubitem{Wrote data analysis support tools in Python that parse various vendor log formats (Seegrid motion controller logs, Kollmorgen Vehicle Diagnostic Tool traces and Kollmorgen NDC8 blackboxes, CANBus logs) to facilitate troubleshooting during development.}
		\ressubitem{Leveraged data analysis tools and experience to assist guidance system vendor with retuning motion control parameters for Raymond's 3030 AutoStacker AGV, facilitating higher operating speeds and improved path accuracy.}
		\ressubitem{Developed safety controller and safety LiDAR architecture for Raymond's next-generation AGVs using SICK MicroScan3 LiDARs integrated with SICK FlexiSoft safety controllers using SICK EFI-Pro}
		\ressubitem{Rewrote Raymond's Matlab-based second-generation safety LiDAR field generation tool using Python and the open source geometry library Shapely, greatly reducing time needed to regenerate LiDAR configurations during vehicle development}
		\ressubitem{Lead engineer for development of Raymond's In-Aisle Detection system, an operator assistance/training aid intended to reduce collision occurrences in warehouses}
		\ressubitem{Developed initial prototype of the successor to IADS which integrated more tightly with the vehicle, leveraging raw LiDAR data processing to send a speed limit to the vehicle over CAN.}
		\ressubitem{Developed prototype of a remote operation system for forklifts, using Robot Operating System (ROS) to translate PlayStation controller inputs to CAN messages.}
		\ressubitem{Implemented a VPN-based "Virtual Cable" solution using OpenWRT and OpenVPN to allow remote diagnostics of Raymond's automated products without requiring a physical cable connection to a moving vehicle.  These systems also provided CAN logging capability that was vastly more economical than Vector's product line, allowing wider deployment of CAN loggers on development vehicles.}
		\ressubitem{Worked with software development teams to modernize our software build processes:  Using Docker to build and deploy to embedded Linux systems, and transitioning from a proprietary build system to CMake that facilitated continuous integration. }
	\end{ressubsec}

	\begin{ressubsec}{Lockheed Martin Corporation}{December 2005 - July 2016, promotion from Systems Engineer to Systems Engineer Senior in 2011}
        \ressubitem{Display test and measurement subject matter expert - sunlight readability evaluation, night vision compatibility evaluation, contrast and luminance measurement, chromaticity measurement in support of multiple platforms}
	  	\ressubitem{Troubleshot Navy fleet issue at customer location involving APN-194 radar altimeter's ability to interface with MH-60R helicopter}
        \ressubitem{Leveraged embedded Linux experience to assist vendor with integrating the Advanced Data Transfer System with the MH-60R, including identifying solutions to severe performance issues.  Key participant in weekly teleconferences with Navy customer and subsystem vendor - presented potential solutions to customer/vendor on a routine basis}
        \ressubitem{Routinely relied upon to troubleshoot network interfacing issues on multiple aircraft platforms using Wireshark and knowledge of Ethernet and TCP/IP, including identifying whether issues were hardware or software or a combination of both.}
        \ressubitem{Maintained a custom toolset written in Python to parse and analyze flight data logs from helicopters}
        \ressubitem{Multiple years of anti-tamper systems engineering experience on multiple platforms, including frequent customer presentations}

        \ressubitem{RF systems engineer, including EMI/EMC testing of the MH-60R aircraft at the systems level, supervising MIL-STD-461 and MIL-STD-464A EMI testing, working with vendors to resolve failures with techniques such as improved shielding or grounding of cable harnesses, and implementing a test signal generator for Automatic Identification System (AIS) receiver testing}
	\end{ressubsec}

	\begin{ressubsec}{Andrew Corporation RF Power Amplifier Group}{May 2002 to Sept. 2003}
		\ressubitem{Developed new amplifier architectures}
	    \ressubitem{Automated UMTS linear power amplifier testing}
		\ressubitem{Created custom test fixtures and Perl scripts to interface with test equipment via GPIB in addition to using LabView testing suites}
		\ressubitem{Familiar with test equipment including Rohde \& Schwarz spectrum analyzers, HP signal generators, power meters, power supplies, and network analyzers}
		\ressubitem{Demonstrated a new amplifier architecture suitable to potential customers for the upcoming iBiquity IBOC digital radio broadcasting standard. The new design was 50\% more efficient than conventional designs.}
		\ressubitem{Co-inventor of an amplifier architecture patent related to extremely high-efficiency linear RF amplifiers (US 6930547 B2)}
	\end{ressubsec}


\end{ressection}

\begin{ressection}{Skills}

	\resitem{\textbf{Operating Systems:} Linux (Ubuntu, embedded custom distributions, Android), Windows 98/2000/XP/7/10/11, VxWorks 5.x/6.x}

	\begin{reslist}{\textbf{Computer Languages:}}
	  	\ressubitem{Proficient in Matlab, C, Python, and Perl, including embedding C into Perl}
        \ressubitem{Proficient at using git for source code management and Gerrit for source change review}
	  	\ressubitem{Familiar with Java, Node.js, OpenCL, Structured Text, and C++}
	\end{reslist}

	\resitem{\textbf{Foreign Languages:} German (5 years)}

\end{ressection}

\begin{ressection}{Interests and Activities}
		\resitem{Amateur radio (including antenna design and construction)}
		\resitem{Electronics design (LED circuitry, microcontrollers, and audio circuit design)}
		\resitem{Photography and videography, including creating an HDMI livestreaming system using a Raspberry Pi}
	    \resitem{Android platform maintenance and bringup.  Worked with a global team of volunteers across multiple timezones to bring new versions of Android to cell phones not supported by the manufacturer.  The project involved a great deal of reverse engineering of undocumented interface changes made by vendors to low-level interfaces that supported audio, camera, display hardware and sensors for touch, position and acceleration.  Coordinated bringup for four years of Android releases across a variety of phones from multiple manufacturers.}
        \resitem{Currently reverse engineering Sony E-mount lens control protocols for the purposes of adapting other lenses to the camera or adding autofocus to manual lenses, using a logic analyzer and self-authored Python data analysis tools}
\end{ressection}

\begin{ressection}{Citizenship and Security Clearance}

	\resitem{United States citizen}

	\resitem{Security clearance information available upon request}

\end{ressection}

\end{document}



