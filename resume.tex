% Andrew Dodd's Resume
% Created: 12 Jan 2005
% $Date: 2005-02-20 19:38:05 -0500 (Sun, 20 Feb 2005) $
% $Rev: 7 $

\documentclass[10pt,oneside]{article}
\usepackage{geometry}
\usepackage[T1]{fontenc}

\pagestyle{empty}
\geometry{letterpaper,tmargin=1in,bmargin=1in,lmargin=1in,rmargin=1in,headheight=0in,headsep=0in,footskip=.3in}

\setlength{\parindent}{0in}
\setlength{\parskip}{0in}
\setlength{\itemsep}{0in}
\setlength{\topsep}{0in}
\setlength{\tabcolsep}{0in}

% Name and contact information
\newcommand{\name}{Andrew Thomas Dodd}
\newcommand{\addra}{241 Hickories Park Road, Apt. D6}
\newcommand{\addrb}{Owego, NY 13827}
\newcommand{\addr}{\addra, \addrb}
\newcommand{\phone}{(607) 759-0996}
\newcommand{\email}{atd7@cornell.edu}

\hyphenation{Summer}

\makeatletter
\def\txtname{resume_plain}\edef\txtname{\expandafter\strip@prefix\meaning\txtname}
\edef\fname{\jobname}

\ifx\fname\txtname
%%%%%%%%%%%%%%%%%%%%%%%%%%%%%%%%%%%%%%%%%%%%%%%%%%%%%%%%%
% New commands and environments
% Stripped down version for text-only output

% This defines how the name looks
\newcommand{\bigname}[1]{
	\begin{center}\Huge\scshape#1\end{center}
}

% A ressection is a main section (<H1>Section</H1>)
\newenvironment{ressection}[1]{
	\vspace{4pt}
	{\Large#1}
	\begin{itemize}
	\vspace{3pt}
}{
	\end{itemize}
}

% A resitem is a simple list element in a ressection (first level)
\newcommand{\resitem}[1]{
	\vspace{-4pt}
	\item \begin{flushleft} #1 \end{flushleft}
}

% A ressubitem is a simple list element in anything but a ressection (second level)
\newcommand{\ressubitem}[1]{
	\vspace{-1pt}
	\item \begin{flushleft} #1 \end{flushleft}
}

% A resbigitem is a complex list element for stuff like jobs and education:
%  Arg 1: Name of company or university
%  Arg 2: Location
%  Arg 3: Title and/or date range
\newcommand{\resbigitem}[3]{
	\vspace{-5pt}
	\item
	\textbf{#1}---#2; \quad \textit{#3}
}

% This is a list that comes with a resbigitem
\newenvironment{ressubsec}[3]{
	\resbigitem{#1}{#2}{#3}
	\vspace{-2pt}
	\begin{itemize}
}{
	\end{itemize}
}

% This is a simple sublist
\newenvironment{reslist}[1]{
	\resitem{\textbf{#1}}
	\vspace{-5pt}
	\begin{itemize}
}{
	\end{itemize}
}


\else
%%%%%%%%%%%%%%%%%%%%%%%%%%%%%%%%%%%%%%%%%%%%%%%%%%%%%%%%%
% New commands and environments

% This defines how the name looks
\newcommand{\bigname}[1]{
	\begin{center}\fontfamily{phv}\selectfont\Huge\scshape#1\end{center}
}

% A ressection is a main section (<H1>Section</H1>)
\newenvironment{ressection}[1]{
	\vspace{4pt}
	{\fontfamily{phv}\selectfont\Large#1}
	\begin{itemize}
	\vspace{3pt}
}{
	\end{itemize}
}

% A resitem is a simple list element in a ressection (first level)
\newcommand{\resitem}[1]{
	\vspace{-4pt}
	\item \begin{flushleft} #1 \end{flushleft}
}

% A ressubitem is a simple list element in anything but a ressection (second level)
\newcommand{\ressubitem}[1]{
	\vspace{-1pt}
	\item \begin{flushleft} #1 \end{flushleft}
}

% A resbigitem is a complex list element for stuff like jobs and education:
%  Arg 1: Name of company or university
%  Arg 2: Location
%  Arg 3: Title and/or date range
\newcommand{\resbigitem}[2]{
	\vspace{-5pt}
	\item
	\textbf{#1}---\textit{#2}
}

% This is a list that comes with a resbigitem
\newenvironment{ressubsec}[2]{
	\resbigitem{#1}{#2}
	\vspace{-2pt}
	\begin{itemize}
}{
	\end{itemize}
}

% This is a simple sublist
\newenvironment{reslist}[1]{
	\resitem{\textbf{#1}}
	\vspace{-5pt}
	\begin{itemize}
}{
	\end{itemize}
}


\fi
%%%%%%%%%%%%%%%%%%%%%%%%%%%%%%%%%%%%%%%%%%%%%%%%%%%%%%%%%
% Now for the actual document:

\begin{document}

\ifx\fname\txtname
\else
\fontfamily{ppl} \selectfont
\fi

% Name with horizontal rule
\bigname{\name}

\ifx\fname\txtname
\else
\vspace{-8pt} \rule{\textwidth}{1pt}
\fi

\ifx\fname\txtname
\vspace{-1pt} {\small\itshape \addra

\addrb

\phone

\email}

\else
\vspace{-1pt} {\small\itshape \addr \hfill \phone; \email}
\fi

\vspace{8 pt}


%%%%%%%%%%%%%%%%%%%%%%%%%
\begin{ressection}{Objective}
	\resitem{Currently seeking employment in an RF design or wireless communications field}
\end{ressection}

\begin{ressection}{Education}

	\begin{ressubsec}{Rutgers University}{M.S. in Electrical and Computer Engineering}
		\ressubitem{Graduated in December 2005}
		\ressubitem{Concentration in Digital Signal Processing and Wireless Communications}
	\end{ressubsec}

	\begin{ressubsec}{Cornell University College of Engineering}{Bachelor of Science in Electrical Engineering}
		\ressubitem{Graduated in May 2002: John McMullen Dean's Scholar}
		\ressubitem{Concentration in RF Design and Wireless Communications}
	\end{ressubsec}

\end{ressection}

\begin{ressection}{Work Experience}

	\begin{ressubsec}{Johns Hopkins University Center for Talented Youth}{Summer 2005}
		\ressubitem{Worked as a teaching assistant for two of CTY's courses, Data Structures and Algorithms and Fundamentals of Computer Science.  CTY is a summer program for gifted high school students run by JHU at sites throughout the country.}
	\end{ressubsec}

	\begin{ressubsec}{IP Options, Inc.}{Summer 2004}
		\ressubitem{Researched patent applicability evaluations for an Intellectual Property firm}
	\end{ressubsec}

	\begin{ressubsec}{Andrew Corporation RF Power Amplifier Group}{May 2002 to Sept. 2003}
		\ressubitem{Developed new amplifier architectures}
	        \ressubitem{Automated UMTS linear power amplifier testing}
		\ressubitem{Created custom test fixtures and Perl scripts to interface with test equipment via GPIB in addition to using LabView testing suites}
		\ressubitem{Familiar with test equipment including Rohde \& Schwarz spectrum analyzers, HP signal generators, power meters, power supplies, and network analyzers}
		\ressubitem{Demonstrated a new amplifier architecture suitable to the upcoming iBiquity IBOC digital radio broadcasting standard. The new design was 50\% more efficient than conventional designs.}
		\ressubitem{Co-inventor of a pending amplifier architecture patent related to extremely high-efficiency linear RF amplifiers (US 2004/0027198 A1)}
	\end{ressubsec}

	\begin{ressubsec}{Cornell University Department of Electrical and Computer Engineering}{Summer 2001}
		\ressubitem{Assisted in the design of a digital receiver for Cornell University's CUPRI radar system using the Analog Devices AD6620 digital downconverter device and a National Instruments PCI-DIO-32HS digital I/O card}
	\end{ressubsec}

	\begin{ressubsec}{Lucent Technologies Silicon Processing Research Group}{Summer 2000}
		\ressubitem{Independently created a CGI front-end to a simulation tool and software to process an ion implantation simulator output. Independently initiated a port of simulation tools from older Unix variants to Linux so that extensive simulation could be done on a low end machine to meet deadlines.}
	\end{ressubsec}

	\begin{ressubsec}{Cornell University Department of Electrical and Computer Engineering}{Spring 2000 to September 2002}
		\ressubitem{Part-time student research in Cornell University's Digital Signal Compression and Video Encoding Research (DISCOVER) lab}
		\ressubitem{Assisted in testing of the lab's CU30 algorithm on Linux systems, and implementing MMX optimizations in some portions of the algorithm}
	\end{ressubsec}

	\begin{ressubsec}{Cornell University Campus Store}{Summer 1999 to Spring 2000}
		\ressubitem{Sales representative in the store's computer sales and repair department}
	\end{ressubsec}

	\begin{ressubsec}{Lucent Technologies}{Summer 1997, Summer 1998}
		\ressubitem{Internship in Lucent's Competitive Analysis and Reverse Engineering group.  (Business Communications Systems unit, now Avaya)}
	\end{ressubsec}

	\begin{ressubsec}{Bridgewater-Raritan High School}{Fall 1996 to Spring 1998}
		\ressubitem{Volunteer work as system administrator and co-webmaster of the school's Linux-based web and email servers}
	\end{ressubsec}

\end{ressection}

\begin{ressection}{Skills}

	\resitem{\textbf{Operating Systems:} Linux (Gentoo and Red Hat), Windows 98/2000/XP}

	\begin{reslist}{Computer Languages:}

		\ressubitem{Proficient in Matlab and Perl}

		\ressubitem{Familiar with Python, Java, C++, and C, including embedding C code into Perl programs}

	\end{reslist}

	\resitem{\textbf{Foreign Languages:} German (5 years)}

\end{ressection}

\begin{ressection}{References}

	\resitem{Dr. Joseph Lipowski - VP of Research, Andrew Corporation - \textit{(908) 546-4610}}

	\resitem{Dr. Rajiv Chandrasekaran - Senior Engineer, Andrew Corporation - \textit{(908) 546-4870}}

	\resitem{Dr. Toby Berger - Professor of Engineering, Cornell University - \textit{(607) 255-1447}}

\end{ressection}

\begin{ressection}{Interests and Activities}

	\resitem{\textbf{Interests:} Amateur radio (including antenna design and construction), electronics (LED circuitry, microcontrollers, and audio circuit design), photography}

	\resitem{\textbf{Extracurricular Activities - Undergraduate:} Marching Band, Pep Band, Cornell Student Linux Users Group, Secretary of the Cornell Amateur Radio Club in the 2001-2002 academic year}

	\resitem{\textbf{Extracurricular Activities - Graduate:} Rutgers Student Linux Users Group, Rutgers Off Campus Students Association}

\end{ressection}

\begin{ressection}{Citizenship}

	\resitem{United States}

\end{ressection}

\end{document}



