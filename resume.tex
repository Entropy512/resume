% Andrew Dodd's Resume
% Created: 12 Jan 2005
% $Date: 2005-02-20 19:38:05 -0500 (Sun, 20 Feb 2005) $
% $Rev: 7 $

\documentclass[10pt,oneside]{article}
\usepackage{geometry}
\usepackage[T1]{fontenc}
\usepackage{enumitem}

\pagestyle{empty}
\geometry{letterpaper,tmargin=0.7in,bmargin=0.7in,lmargin=0.7in,rmargin=0.7in,headheight=0in,headsep=0in,footskip=.3in}

\setlength{\parindent}{0in}
\setlength{\parskip}{0in}
\setlength{\itemsep}{0in}
\setlength{\topsep}{0in}
\setlength{\tabcolsep}{0in}

% Name and contact information
\newcommand{\name}{Andrew Thomas Dodd}
\newcommand{\addra}{264 Lower Stella Ireland Road, Apt 1A}
\newcommand{\addrb}{Binghamton, NY 13905}
\newcommand{\addr}{\addra, \addrb}
\newcommand{\phone}{(607) 759-0996}
\newcommand{\email}{atdodd@gmail.com}

\hyphenation{Summer}

\makeatletter
\def\txtname{resume_plain}\edef\txtname{\expandafter\strip@prefix\meaning\txtname}
\edef\fname{\jobname}

\ifx\fname\txtname
%%%%%%%%%%%%%%%%%%%%%%%%%%%%%%%%%%%%%%%%%%%%%%%%%%%%%%%%%
% New commands and environments
% Stripped down version for text-only output

% This defines how the name looks
\newcommand{\bigname}[1]{
	\begin{center}\Huge\scshape#1\end{center}
}

% A ressection is a main section (<H1>Section</H1>)
\newenvironment{ressection}[1]{
	\vspace{4pt}
	{\Large#1}
	\begin{itemize}
	\vspace{3pt}
}{
	\end{itemize}
}

% A resitem is a simple list element in a ressection (first level)
\newcommand{\resitem}[1]{
	\vspace{-4pt}
	\item \begin{flushleft} #1 \end{flushleft}
}

% A ressubitem is a simple list element in anything but a ressection (second level)
\newcommand{\ressubitem}[1]{
	\vspace{-1pt}
	\item \begin{flushleft} #1 \end{flushleft}
}

% A resbigitem is a complex list element for stuff like jobs and education:
%  Arg 1: Name of company or university
%  Arg 2: Location
%  Arg 3: Title and/or date range
\newcommand{\resbigitem}[3]{
	\vspace{-5pt}
	\item
	\textbf{#1}---#2; \quad \textit{#3}
}

% This is a list that comes with a resbigitem
\newenvironment{ressubsec}[3]{
	\resbigitem{#1}{#2}{#3}
	\vspace{-2pt}
	\begin{itemize}
}{
	\end{itemize}
}

% This is a simple sublist
\newenvironment{reslist}[1]{
	\resitem{\textbf{#1}}
	\vspace{-5pt}
	\begin{itemize}
}{
	\end{itemize}
}


\else
%%%%%%%%%%%%%%%%%%%%%%%%%%%%%%%%%%%%%%%%%%%%%%%%%%%%%%%%%
% New commands and environments

% This defines how the name looks
\newcommand{\bigname}[1]{
	\begin{center}\fontfamily{phv}\selectfont\Huge\scshape#1\end{center}
}

% A ressection is a main section (<H1>Section</H1>)
\newenvironment{ressection}[1]{
	\vspace{4pt}
	{\fontfamily{phv}\selectfont\Large#1}
	\begin{itemize}[leftmargin=12pt]
	\vspace{3pt}
}{
	\end{itemize}
}

% A resitem is a simple list element in a ressection (first level)
\newcommand{\resitem}[1]{
	\vspace{-4pt}
	\item \begin{flushleft} #1 \end{flushleft}
}

% A ressubitem is a simple list element in anything but a ressection (second level)
\newcommand{\ressubitem}[1]{
	\vspace{-1pt}
	\item \begin{flushleft} #1 \end{flushleft}
}

% A resbigitem is a complex list element for stuff like jobs and education:
%  Arg 1: Name of company or university
%  Arg 2: Location
%  Arg 3: Title and/or date range
\newcommand{\resbigitem}[2]{
	\vspace{-5pt}
	\item
	\textbf{#1}---\textit{#2}
}

% This is a list that comes with a resbigitem
\newenvironment{ressubsec}[2]{
	\resbigitem{#1}{#2}
	\vspace{-2pt}
	\begin{itemize}[leftmargin=12pt]
}{
	\end{itemize}
}

% This is a simple sublist
\newenvironment{reslist}[1]{
	\resitem{\textbf{#1}}
	\vspace{-5pt}
	\begin{itemize}[leftmargin=12pt]
}{
	\end{itemize}
}


\fi
%%%%%%%%%%%%%%%%%%%%%%%%%%%%%%%%%%%%%%%%%%%%%%%%%%%%%%%%%
% Now for the actual document:

\begin{document}

\ifx\fname\txtname
\else
\fontfamily{ppl} \selectfont
\fi

% Name with horizontal rule
\bigname{\name}

\ifx\fname\txtname
\else
\vspace{-8pt} \rule{\textwidth}{1pt}
\fi

\ifx\fname\txtname
\vspace{-1pt} {\small\itshape \addra

\addrb

\phone

\email}

\else
\vspace{-1pt} {\small\itshape \addr \hfill \phone; \email}
\fi

\vspace{8 pt}


%%%%%%%%%%%%%%%%%%%%%%%%%
\begin{ressection}{Objective}
	\resitem{Currently seeking employment in electronics hardware design and/or embedded software}
\end{ressection}

\begin{ressection}{Education}

	\begin{ressubsec}{Rutgers University}{M.S. in Electrical and Computer Engineering, December 2005}
		\ressubitem{Concentration in Digital Signal Processing and Wireless Communications}
	\end{ressubsec}

	\begin{ressubsec}{Cornell University College of Engineering}{Bachelor of Science in Electrical Engineering, May 2002}
		\ressubitem{Concentration in RF Design and Wireless Communications}
	\end{ressubsec}

\end{ressection}

\begin{ressection}{Work Experience}

	\begin{ressubsec}{Lockheed Martin Corporation}{December 2005 - July 2016, promotion from Systems Engineer to Systems Engineer Senior in 2011}
          \ressubitem{Display test and measurement subject matter expert - sunlight readability evaluation, night vision compatibility evaluation, contrast and luminance measurement, chromaticity measurement in support of multiple platforms}
	  \ressubitem{Troubleshot Navy fleet issue at customer location involving APN-194 radar altimeter's ability to interface with MH-60R helicopter}
          \ressubitem{Leveraged embedded Linux experience to assist vendor with integrating the Advanced Data Transfer System with the MH-60R, including identifying solutions to severe performance issues.  Key participant in weekly teleconferences with Navy customer and subsystem vendor - presented potential solutions to customer/vendor on a routine basis}
          \ressubitem{Routinely relied upon to troubleshoot network interfacing issues on multiple aircraft platforms using Wireshark and knowledge of Ethernet and TCP/IP, including identifying whether issues were hardware or software or a combination of both.}
          \ressubitem{Performed analysis of each gain stage of the MH-60R Audio Management Computer using an Audio Precision AP2700 audio analyzer to determine optimal gain settings to improve headset volume while minimizing distortion}
          \ressubitem{Maintained a custom toolset written in Python to parse and analyze flight data logs from helicopters}
          \ressubitem{Worked on tools to preprocess OpenStreetMap data to evaluate feature-matching algorithms for non-GPS precision navigation, for the purposes of guiding a vehicle using imaging sensor data when GPS is not available}
          \ressubitem{Multiple years of anti-tamper systems engineering experience on multiple platforms, including frequent customer presentations}
          \ressubitem{Characterized and integrated beyond-line-of-sight (BLOS) satellite datalink for remote control of K-MAX autonomous helicopter}
          \ressubitem{RF systems engineer, including EMI/EMC testing of the MH-60R aircraft at the systems level, supervising MIL-STD-461 and MIL-STD-464A EMI testing, working with vendors to resolve failures with techniques such as improved shielding or grounding of cable harnesses, and implementing a test signal generator for Automatic Identification System (AIS) receiver testing}
	\end {ressubsec}

	\begin{ressubsec}{Johns Hopkins University Center for Talented Youth}{Summer 2005}
		\ressubitem{Worked as a teaching assistant for two of CTY's courses, Data Structures and Algorithms and Fundamentals of Computer Science.  CTY is a summer program for gifted high school students run by JHU at sites throughout the country.}
	\end{ressubsec}

	\begin{ressubsec}{IP Options, Inc.}{Summer 2004}
		\ressubitem{Researched patent applicability evaluations for an Intellectual Property firm}
	\end{ressubsec}

	\begin{ressubsec}{Andrew Corporation RF Power Amplifier Group}{May 2002 to Sept. 2003}
		\ressubitem{Developed new amplifier architectures}
	        \ressubitem{Automated UMTS linear power amplifier testing}
		\ressubitem{Created custom test fixtures and Perl scripts to interface with test equipment via GPIB in addition to using LabView testing suites}
		\ressubitem{Familiar with test equipment including Rohde \& Schwarz spectrum analyzers, HP signal generators, power meters, power supplies, and network analyzers}
		\ressubitem{Demonstrated a new amplifier architecture suitable to potential customers for the upcoming iBiquity IBOC digital radio broadcasting standard. The new design was 50\% more efficient than conventional designs.}
		\ressubitem{Co-inventor of an amplifier architecture patent related to extremely high-efficiency linear RF amplifiers (US 6930547 B2)}
	\end{ressubsec}

	\begin{ressubsec}{Cornell University}{Multiple Employment Periods}
	  \ressubitem{Summer 2001 - Assisted in the design of a digital receiver for Cornell University's CUPRI radar system using the Analog Devices AD6620 digital downconverter device and a National Instruments PCI-DIO-32HS digital I/O card}
	  \ressubitem{Spring 2000 onwards through graduation - Assisted in testing of the Digital Signal Compression and Video Encoding Research (DISCOVER) Lab's CU30 algorithm on Linux systems, and implementing MMX optimizations in some portions of the algorithm}
          \ressubitem{Summer 1999 to Spring 2000 - Sales representative in the campus store's computer sales and repair department}
	\end{ressubsec}

	\begin{ressubsec}{Lucent Technologies}{Summers of 1997, 1998, 2000}
	  \ressubitem{2000 (Silicon Processing Research Group) Independently created a CGI front-end to a simulation tool and software to process an ion implantation simulator output. Independently initiated a port of simulation tools from older Unix variants to Linux so that extensive simulation could be done on a low end machine to meet deadlines.}
          \ressubitem{Summer 1997, 1998 (Competitive Analysis and Reverse Engineering group, Business Communications Systems unit, now Avaya) - Worked on identifying potential patent infringement and cost/feature/usability analysis of competitor's products}
	\end{ressubsec}

\end{ressection}

\begin{ressection}{Skills}

	\resitem{\textbf{Operating Systems:} Linux (Ubuntu, embedded custom distributions, Android), Windows 98/2000/XP/7/10/11, VxWorks 5.x/6.x}

	\begin{reslist}{\textbf{Computer Languages:}}
	  \ressubitem{Proficient in Matlab, C, Python, and Perl, including embedding C into Perl}
          \ressubitem{Proficient at using git for source code management and Gerrit for source change review}
	  \ressubitem{Familiar with Java, Node.js and C++}
	\end{reslist}

	\resitem{\textbf{Foreign Languages:} German (5 years)}

\end{ressection}

\begin{ressection}{Interests and Activities}
		\resitem{Amateur radio (including antenna design and construction)}
		\resitem{Electronics design (LED circuitry, microcontrollers, and audio circuit design)}
		\resitem{Photography and videography, including creating an HDMI livestreaming system using a Raspberry Pi}
	        \resitem{Android platform maintenance and bringup.  Worked with a global team of volunteers across multiple timezones to bring new versions of Android to cell phones not supported by the manufacturer.  The project involved a great deal of reverse engineering of undocumented interface changes made by vendors to low-level interfaces that supported audio, camera, display hardware and sensors for touch, position and acceleration.  Coordinated bringup for four years of Android releases across a variety of phones from multiple manufacturers.}
                \resitem{Currently reverse engineering Sony E-mount lens control protocols for the purposes of adapting other lenses to the camera or adding autofocus to manual lenses, using a logic analyzer and self-authored Python data analysis tools}
		\resitem{Executive board member of Southern Tier Young Professionals from 2010 to 2014}
\end{ressection}

\begin{ressection}{Citizenship and Security Clearance}

	\resitem{United States citizen}

	\resitem{Security clearance information available upon request}

\end{ressection}

\end{document}



